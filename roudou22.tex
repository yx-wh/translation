\documentclass[a4paper,12pt]{article}
\usepackage{fontspec,xunicode,xltxtra}
\usepackage{booktabs}
\usepackage{indentfirst}
\usepackage{graphicx} 
\usepackage[SlantFont]{xeCJK}

\setCJKfamilyfont{mincho}{TakaoExMincho}
\setCJKfamilyfont{gothic}{TakaoExGothic}
\newcommand\ja[1]{{\CJKfamily{gothic}#1}}
\setCJKmainfont[BoldFont=WenQuanYi Micro Hei]{Adobe Song Std}
\setCJKsansfont{WenQuanYi Micro Hei}

\setmainfont[Mapping=tex-text]{Liberation Serif}
\setsansfont{Liberation Sans Narrow}
\setmonofont{Liberation Mono}
\XeTeXlinebreaklocale "zh"
\XeTeXlinebreakskip = 0pt plus 1pt minus 0.1pt

\linespread{1.3}
\usepackage[top=25mm,bottom=20mm,left=30mm,right=30mm]{geometry}

\renewcommand{\contentsname}{目录}


\author{yx-wh}
\title{roudou22}
\begin{document}

\newpage
日本劳动年鉴 战后特集 第22集
\setcounter{tocdepth}{3}
\tableofcontents
\newpage

\section{劳资争议}

\subsection{争议总体情况}

\subsubsection{概况}

日本的工人运动萌芽于明治末年,第一次世界大战后呈现出活跃的态势,但是组织率(工会会员占全体劳动者的比例)最高也不过是昭和六年(1931)九一八事变前夕的7.9\%,绝大部分劳动者并未参与工会组织。在众多弹压法规下,工人运动中仅存的那些合法的部分也在昭和十二、十三年(1937、1938年)之后,也就是进入完全战时体制之后被强力取缔,跟工人运动沾的上光的一切东西大概都被有关部门给完全控制了。当然,对于此番压制,工人一直留血抵抗直到最后,可总体而言,工人的自主运动已被昭和十五年(1940年)第二次近卫内阁从上至下所组织的「产业报国会组织」所代替。

在此种窒息状态中的日本的工人运动随着战败逐渐复苏。昭和二十年(1945年)八月十五日日本所接受的《波茨坦公告》中有「日本政府必将阻止日本人民民主趋势之复兴及增强之所有障碍予以消除」这一条文;基于此,昭和二十年九月三十日,大日本产业报国会被解散;十月四日,按照对日本政府发出的关于「撤销对政治、信教及民权自由的限制」的备忘录,《治安维持法》和特高警察被废止,在狱中的共产主义者被释放;至此,阻止工会运动自由发展的桎梏被解除。在这之后,全国各地掀起了建立工会的运动高潮。

伴随着战败,日本资本主义体制内的矛盾进一步深化,不得不直面严重的生产危机。为了再组织此种濒临崩溃的资本主义所采取的通胀政策使得人的生活成本急剧上涨,工人生活的穷困极为严重。处于此种困难的生活状态中的工人为了维持生存,必然自发地从事工人运动。因此可以说,虽然战后的工人运动一方面是由国际民主势力所推动的,但另一方面也是日本资本主义的体制性矛盾的必然产物。

但是这种自然发生的工人运动,除了刚战后就受到老一辈工人运动家的指导的个别地方,通常都还是处于混沌的状态。从此中状态中脱离的第一步就是昭和二十年(1945年)十一月二十三日为追究干部的战争责任而奋起的读卖新闻社从业人员的斗争。读卖争议中,劳动者积极地将战犯整肃(这是战败后最大的问题之一),并将业务管理作为斗争的手段,这两点上读卖争议是划时代的。

在斗争的具体目标和斗争方法上,读卖争议对混沌的工人运动有很多启发;但是成立的工会总是缺乏有组织性的联络,仍是单独的工会。此外从此时开始渐渐活跃起来的劳资争议的要求内容仍旧停留于简单的承认工会和涨工资等等,斗争手段也太naive。此外,在战争中劳动阶级积累的不满一时间得以爆发的实例也不容忽视——北海道三菱美呗煤矿出现的所谓「人民审判事件」正是其典型。当然,尽管日本的工人运动在重新出发时有着「广大群众长时间受到压迫而落后」的不利条件,此时还处于这样不成熟的阶段,但是它在消解过去处于强有力支配地位的生产关系上起到的重要作用不容否定。但与此同时战后的工人运动需要解决生产崩溃这一新的问题。为此采用的积极的解决方式就是「生产管理」(业务管理)。读卖争议时就已出现的生产管理在随后的京成电气铁道的争议(昭和二十年(1945年)十二月)中得到继承和更彻底地发扬。京成电铁的生产管理在贯彻(劳方)要求的同时显著增强了工作效率。这给了面临生产崩溃这一现实的工人运动极大的启发。生产管理是承担生产的劳动阶级积极地形成的斗争战术;在打破资本家的关厂策略的同时,回应了社会上对重开生产的需要。昭和二十一年(1946年)三月~六月,以生产管理作为斗争方式的争议次数渐渐增多,数量远超罢工怠工——这展现了战后初期工人运动的特点。

最初各个工会都是毫无联系独自争斗的,但是从昭和二十年(1945年)十二月左右开始,在三菱重工业东京机器制作所、东京芝浦电气川崎地区等地,同一资方下的工厂或是临近的工厂开始共同斗争了。除了像这样通过共同斗争让组织自由成长的,也有人开始有意识地将同一地区的各个单独工会联合起来。昭和二十年(1945年)十二月二十五日,召开了神奈川县第一次工厂代表者会议。神奈川县下二十一座工厂的代表相聚一堂,决议开展共同斗争。之后在东京的城南、城东、城西等地区也有「协议会」成立。经过共同斗争而诞生的这些「协议会」成为有组织的共同斗争的指导机关。「神奈川县劳动组合协议会」指导了东芝川崎地区六工厂和日本钢管鹤见工厂etc的争议;「城南地区协议会」指导了日本起重机、电业社\footnote{译者:此处原文为「牡」,似乎是错别字。}的争议;双方作为地方上的指导团体都取得了很大成果。

在地域性工会的基础上形成了更大规模的地域性工会,「关东地方劳动组合协议会」。这个机关在「行业整理」中起到了过渡性指导机关的作用。渐渐再编成为全国性的单一行业工会后,昭和二十一年(1946年)二月二十日,在最早完成「行业整理」的「日本新闻通信工会」的倡导下,召开了「全国行业工会会议准备会」的发起人会。在组织上有了上述发展的工人运动,在昭和二十一年(1946年)二月、三月虽出现了全面的退潮,但随着四月七日的打到币原内阁人民大会和五月的国际劳动节的到来,运动再次高涨,工会的经济主义斗争带上了政治斗争的色彩。以「全国电影从业员组合同盟」的共同斗争为契机,斗争的形态发生了质的变化,同一行业中属于不同资本的企业展开了大规模的共同斗争。在「全国电影从业员组合同盟」之后,「全日本印刷出版工会」「日本发送电从业员工会」「日本钢铁工会」「全日本机械器具工会」等纷纷开始了共同斗争。这些各个行业的共同斗争最后导致六月二十六日的「行业工会准备大会」\footnote{日文:「\ja{産業別労働組合準備大会}」。}的召开,在审议了当下斗争方针、组织方针、劳动战线的统一方针之后,八月十九日,成立了「行业工会会议」\footnote{日文简称:「\ja{産別会議}」。全称:「\ja{全日本産業別労働組合会議}」。维基百科介绍:1946年由21个行业工会(工会会员163万)结成的全国协议体组织。遵循世界工联的基本纲领,在二战后的工会建设期的活动中,和右派的「日本工会总同盟」(「\ja{日本労働組合総同盟}」)相对抗。指导了1946年的十月斗争和1947年的二一大罢工,但在1948年右派的「行业工会民主化同盟」(「\ja{産別民主化同盟}」)成立后,显出颓势,于1958年解散。}。「行业工会会议」的成立标志着劳动阶级的成长,工会运动从经济范畴发展到政治范畴。

先前,在昭和二十年(1945年)十月十日,以过去的日本劳动总同盟、日本工会同盟、日本工会全国评议会等的领导者为中心的「单一工会结成恳谈会」建立了总同盟;在昭和二十一年(1946年)一月,正式定名为「日本工会总同盟」。这样形成的「总同盟」和通过共同斗争组织起来的「行业工会会议」处于对立状态,尽管后来「统一劳动战线关照人会」等多次尝试将两组织合并起来,但它们仍然保持了在劳工战线上各据一方的状态。

如上所述,随着组织的大发展,工会的要求也提高了,除了涨工资外,还增加了确立集体谈判、参与经营的要求。

和工人运动的飞速的活跃化相比,资本家阶级的反应要慢的多。但资方也针对工人组织准备了系统的反击,例如在昭和二十一年(1946年)六月十三日的《关于社会秩序的声明》中确立了否定「生产管理」的方针,在六月的读卖争议中动用了警察三百人加以直接的反击。资本家如斯攻势的背后站着的,是基于自由党、进步党两党联合成立的吉田内阁。六月之后,劳资争议略显颓势,但随着两大工会组织(「行业工会会议」和「总同盟」)的建立又受到激励,九月十月在不断的罢工下展开了大规模的斗争。此即「行业工会会议」的「十月攻势」。当然,像十月攻势这样的大规模共同斗争只有在有了组织的力量之后才是可能的,但使其成为可能的还有更深层的原因。战败后,为了拯救面临体制性危机的日本经济,首先是通过通货膨胀降低了实际工资,其次又大量裁员实现了产业合理化;这两点所导致的工人生活的全面恶化正是使得「十月攻势」这样的大规模斗争成为可能的深层原因。

昭和二十一年(1946年)十月针对海员和国铁两个工会,首先出现了大量裁员。海员和国铁的工会则以反对裁员的罢工对抗。这场斗争是受到失业威胁的劳动阶级的防卫战。海运总局和船舶运营会以「战后日本船舶保有量大量减少」为理由计划裁员六万,对此海员工会展开斗争,不顾工会内的混乱展开了总罢工。

这时候的铁道省也以「即将接纳大陆铁道部门人员十八万、复员者二十万」为口实,计划裁员七万五千。国铁工会总联合决定对此展开总罢工对抗,但在就要开始总罢工前,和当局妥协,整顿计划也被取消。九月以海员和国铁为中心的反裁员斗争,是防卫战,也是十月的广泛共同斗争的先驱。

拉开十月斗争的序幕的,是东京芝浦电气株式会社工会协议会的斗争。在东芝,劳方提出了最低工资制、召开产业会议等要求,但是与资方的交涉最终破裂,于是从十月一日开始一起进入总罢工。在东芝之后,全国煤矿工会、全日本新闻通信工会、日本电器产业工会协议会、全日本机械器具工会、全日本电气工业工会、日本电影戏剧工会等纷纷展开斗争,争议的规模比过去大到不知哪里去了。不光是参加了罢工人劳动者人数有了极大的增长,其斗争状态的组织,以及要求内容的方面也有了极大的成长。在十月斗争中,「行业工会会议」在总罢工中起到了核心领导作用;在要求中也充分反映了产业单一工会下签订统一集体合同的要求,以及复兴生产的积极意欲。与九月斗争反对裁员的自我防卫色彩相比,十月斗争反映了劳动者阶级正是日本复兴的先导,在这个意思上,十月斗争本质上才具有「攻势」的意义。

\subsubsection{争议次数及参加争议的人员}

战后争议的规模和激烈程度是(日本)工人运动史上前所未有的。这既说明了战后政治经济危机的严重程度,也显示了在战时被禁锢的工人阶级现在所能展现的能量。但是,劳动者能量的释放也是经过了曲折的历程的,是一个振幅很大的波动曲线。我们现在通过对争议发生次数和参加人员的分析,揭示其波动历程并阐明战后工人运动的划时代特征。

为方便起见,以昭和二十一年(1946年)前半为第一期,后半为第二期,昭和二十二年(1947年)前半为第三期,后半为第四期。比较昭和二十一年和昭和二十二年,争议次数前者910次、后者932次,参加人员前者2716235人、后者3093963人,无显著差异。

第一期争议次数406次、参加人员232936人,第二期争议次数504次、参加人员2483299人。第二期的争议次数和参与人数都远远比第一期多。第三期又下降为争议次数375次、参加人员788042人,第四期为557次、3093963人,为四期中最大的。由上可知争议都是下半年更显著。但是光靠此不能正确地把握争议的波动。以下是对每期的详尽分析
\begin{itemize}
\item \textbf{第一期(昭和二十一年(1946年)一月~六月)}
  
  昭和二十年(1945年)十二月的争议次数为42次、参加人员为17919人,到了昭和二十一年(1946年)一月激增至64次、36402人。这说明战败后的工人运动,除了那些一开始就以既有的干部为中心开展活动的以外,其他多数都是从第一阶段的「无意识状态」,通过对「提出『经营与资本分离、追究战犯、工会参与经营管理』等高水平要求的读卖争议」的支持,建立了新闻从业员工会,发展到第二阶段的「从单一工会到行业工会」的。也就是说,直到昭和二十年(1945年)十二月末为止,形式上的重点都在于单一工会的形成。从昭和二十一年(1946年)一月开始,渐渐发展至地域工会(这是向行业工会发展的过渡阶段),随着组织的发展,争议次数和参加人员也飞速增加。一月份高涨的争议的潮流,又随着三月《工会法》的施行,在三月到达77次、74289人,五月则是有劳动节和粮食Mayday\footnote{日文维基百科:「\ja{飯米獲得人民大会}」,亦称「\ja{食糧メーデー}」,是1946年5月19日在日本东京都皇居前举行的、抗议政府拖延粮食配给的集会。随着战后社会主义运动的高涨,参与者25万人,为战后最大。},工人运动最为昂扬,呈现了81次、41614人这一最高数值。在五月,从地方、地区的协议会发展而来,完成了行业单一工会组织的建立的有,新闻通信、国铁、递信\footnote{日文:「\ja{逓信}」。递信省是日本过去存在的一个中央官厅,管辖邮政和通信。日本邮政和日本电信电话(NTT)的前身。}、印刷出版、日本通运、教员、煤矿、电影戏剧、钢铁。五月二十日麦元帅发表了《针对大量暴民的游行和骚乱所发出的警告》\footnote{日文:「\ja{多数の暴民によるデモと騒擾に対し警告を発する旨}」}这一声明,吉田内阁也与之呼应,发表了《关于保障社会秩序的政府声明》\footnote{日文:「\ja{社会秩序保持に関する政府声明}」}。上述高涨的运动潮流,也因此在六月降到了52次、9712人。此时工人运动的低潮,可以认为是资本家的积极反攻(从当时读卖争议中进行的直接镇压可以看出)和与之相呼应的、政府的针对工人运动的活动的结果。

\item \textbf{第二期(昭和二十一年(1946年)七月~十二月)}

  在第一期最后显出显著退潮的运动,在七月、八月又渐渐以共同斗争的形式热闹起来,九月,为了反裁员,国铁和海员工会开始斗争,争议次数、参加人员分别激增为62次、563132人。不过这个高涨不过是大规模的十月攻势的前哨战。十月109次、参与者268653人,是战败以来的最高数字。其中特别是同盟罢工有74次,参加人员170822人,为四期中最高。可见十月份工会的斗争性最高。十一月的发生次数虽少,但是十月斗争的继续,运动的波涛并未衰减。十二月81次、1511968人,可以看到参加人员的显著增加。从十一月到十二月,出现了向全递\footnote{日文:「\ja{全逓}」。日文全称:「\ja{全逓信従業員組合}」。日文维基百科:日本邮政公社工会(日:「\ja{日本郵政公社労働組合}」英:Japan Postal Workers' Union)是1946年以全递信从业员工会(「\ja{全逓信従業員組合}」)为名成立的、存在到2007年的日本的工会。2007年10月22日与全日本邮政工会(「\ja{全日本郵政労働組合}」)合并,现为日本邮政集团工会(日:「\ja{日本郵政グループ労働組合}」英:Japan Postal Group Union)。}、国铁、全官公\footnote{日文全称:「\ja{全官公庁労働組合協議会}」。全官公1946年建立,直至二一大罢工为止都是战后初期的工人运动的核心,但在《政令201号》的法律限制和行政整理中被弱化,随着反共的民主化同盟的抬头和官公劳(日文全称:「\ja{日本官公庁労働組合協議会}」)的产生,全官公于1950年解体。}等的中央劳动委员会申诉的情况,这个萌芽后来发展为二一大罢工这一规模前所未有的共同斗争。
  
\item \textbf{第三期(昭和二十二年(1947年)一月~六月)}

  在一月,为了支持国企工人的斗争,其他行业的工人也集结开展共斗,开展了名副其实的全国规模的共同斗争。这个月发生次数为68次,参加人员514946人。此时,没有参加共斗的单独罢工亦不在少数。但是随着麦克阿瑟元帅发布《二一大罢工禁止令》,工会运动进入调整期,四月又有选举,次数、人数低落至34次、48501人。五月六月虽然上扬,但依旧处于较低水平,说明了工会运动朝「既有深度又有广度」转换之困难。

  五、六月特别明显的是,不伴随争议行为的激增。在四月,不伴随争议行为的11次,到了五月21次,六月21次。这是因为,随着吉田内阁后期到片山内阁时期的资本家体制的整备,资本家阵营的组织整备也得以一步步进行(五月十九日全国经营者联合会创立),给劳动者施加了很大压力。这一时期国际形势的变化给劳资双方带来的微妙精神影响也不容忽视(美国通过取缔共产党的法案并将《塔夫脱-蛤特莱法案》提交表决)。

\item \textbf{第四期(昭和二十二年(1947年)七月~十二月)}

  在第三期后半\footnote{译者:很显然这又是错别字。}所出现的倾向在这一时期更为显著。吉田内阁之后的片山内阁,在制定紧急政策的同时确立了新物价体系,作为其一环,制定了1800圆的基本工资(不同工种的平均工资)。这个政策把劳动者框定在低工资上,确保了资本家的产业利润。另一方面,基于紧急政策的配给计划并没有像政府当初所设想的那么成功,加之步步高的通货膨胀,劳动者穷困甚重。七月、八月的争议次数和参加人员上升明显,九月134次、1098766人,说明这一时期争议的潮水如何汹涌。但是和争议次数之多相比,同盟罢工数为九月49次,仅占全体争议的27\%。此外不伴随争议行为的,九月24次,十二月最高,为57次,表明了此时争议的普遍困难。和立刻进入罢工贯彻要求的战后最初期的争议相比,此时的争议连进入罢工阶段都很困难,要贯彻劳方的要求非常不容易。

\end{itemize}
如上所述,我们通过劳资争议的发生次数和参加人员数目,对战败后的工人运动的潮起潮落有了一个比较明确的观察。但是,争议的规模、斗争性以及作为全体的劳动者的能量etc,如果不通过争议的发生次数和继续次数,是不可能得到正确的分析的。

下面我们通过发生次数和继续次数进行如下分析。

争议次数和参加人数增加最多的是第四期。昭和二十二年(1947年)八月之后,争议次数激增,九月最高,达222次。此外参加人员在十一月达到2110773人,其大小仅次于二月人数。九月以后的人数每个月都突破了百万大关。由上可见第四期争议的规模极大,但是这里需要注意的是,其中伴随同盟罢工、同盟怠工、关闭工厂、事业管理的争议行动在所有争议中所占的比例,显著减少了。同时,不伴随争议行为的和有第三者调停斡旋的数量增加,十二月为202次,最多。这种倾向是第四期的特征,尽管争议的规模大且深入,但是其斗争性和第二期第三期相比有显著差异。亦即在此时期,由于资本家和政府施加的压力增强,斗争变得困难,因而各地自发开展争议的倾向变强,同一时刻开展炙热的共同斗争的场景,此时已经看不到了。因此可以说,虽然争议的规模增大了,但是工人阶级发挥影响阶级间力量对比的能量的能力反而弱化了。在发挥能量这一点上,昭和二十二年(1947年)一月、二月比其他月份高到不知哪里去了。二月为147次、参加人员2114947人。如此多的参加人员,是日本工人运动史上前所未有的。此外平均单次参加人员为一月16582人/次,二月14387人/次,这也是最高。可以说日本工人阶级这个时候能量发挥的最多。

在斗争的激烈程度上最有特点的是是昭和二十一年(1946年)十月。十月争议次数176次,参加人员293459人,和其他时期相比数字并不大,但是176次争议中伴随争议行为的有167次,为战后最高。亦即只有9次争议是不伴随争议行为的。由是可知,这正是「行业工会会议」的攻势最激烈的时候(十月攻势)。

先前提到,战后参加人数最多的是昭和二十二年(1947年)二月的2114947人。这占了有组织劳动者的43.6\%,全劳动者的17\%。有组织劳动者的近半数在同一时期参与斗争,这说明了日本工人阶级的令人惊异的成长。

每个月的「每次争议平均参与人员」数字的变化如下。

\begin{table}
  \begin{tabular}{|c|c|c|}
    \hline
    昭和二十一年(1946年) & 一月 & 578人 \\ 
    \hline
                         & 二月 & 434人 \\ 
    \hline
                         & 三月 & 807人 \\ 
    \hline
                         & 四月 & 559人 \\ 
    \hline
                         & 五月 & 434人 \\ 
    \hline
                         & 六月 & 323人 \\ 
    \hline
                         & 七月 & 335人 \\ 
    \hline
                         & 八月 & 455人 \\ 
    \hline
                         & 九月 & 443人 \\ 
    \hline
                         & 十月 & 1667人 \\ 
    \hline
                         & 十一月 & 1214人 \\ 
    \hline
                         & 十二月 & 11975人 \\ 
    \hline
    昭和二十二年(1947年) & 一月 & 16582人 \\ 
    \hline
                         & 二月 & 14387人 \\ 
    \hline
                         & 三月 & 1177人 \\ 
    \hline
                         & 四月 & 2778人 \\ 
    \hline
                         & 五月 & 1250人 \\ 
    \hline
                         & 六月 & 665人 \\ 
    \hline
                         & 七月 & 519人 \\ 
    \hline
                         & 八月 & 442人 \\ 
    \hline
                         & 九月 & 5213人 \\ 
    \hline
                         & 十月 & 9689人 \\ 
    \hline
                         & 十一月 & 11534人 \\ 
    \hline
                         & 十二月 & 10346人 \\ 
    \hline
  \end{tabular}
  \caption{每次争议平均参与人员}
\end{table}
每次争议平均参与人员的如上推移同时表现了斗争能量的上升和下降。从昭和二十一年(1946年)十月开始积累的斗争能量到了昭和二十二年(1947年)一月、二月最高,之后下降,到昭和二十二年(1947年)七八月见底,九月上升,在十一月、十二月超过一万。

通过上述分析我们了解了争议的高低潮、规模、斗争性、工人阶级的能量。此为战后工人运动飞速成长的统计内容。


  
\end{document}
