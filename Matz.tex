\documentclass[a4paper,12pt]{article}
\usepackage{xltxtra,booktabs}
\usepackage{indentfirst}
\usepackage[SlantFont]{xeCJK}

\setCJKmainfont[BoldFont=WenQuanYi Micro Hei]{Adobe Song Std}
\setCJKsansfont{WenQuanYi Micro Hei}

\setmainfont[Mapping=tex-text]{Times New Roman}
\setsansfont{Arial}
\setmonofont{DejaVu Sans Mono}
\XeTeXlinebreaklocale "zh"
\XeTeXlinebreakskip = 0pt plus 1pt minus 0.1pt

\linespread{1.3}
\usepackage[top=25mm,bottom=20mm,left=30mm,right=30mm]{geometry}

\renewcommand{\contentsname}{目录}

\author{yx-wh}
\title{Matz}
\begin{document}

\newpage
\setcounter{tocdepth}{2}
\tableofcontents
\newpage

\section{第1回 与黑客相遇}
大家好,我是松本行弘。在世间以Ruby编程语言的作者这一名字为人所知,职业为程序员,自称黑客。这次连载我希望能介绍介绍我们这些“黑客”的生态和心理。

\subsection{什么是“黑客”?}

“黑客”指的并不是通过网络入侵他人系统、破解密码的坏家伙——话说,最近很少听到这种错误的用法了呢。

Hacker这个词正如其字面意思所示,指的是“实践hack之道的人”。Hack的本义是“用斧头(之类的)猛砍”的意思,后来意义演变为“搞定了程序”。“搞定”也许会给以一种应付差事的感觉,但其实它的言外之意是说工作完成得快,与好坏无关;优秀的工作可以称之为“hack”,为了应付死线(deadline)纯粹交差的工作也是“hack”。热爱上述工作的黑客,就是那种一坐到电脑前(或者任何可编程的东西前)就沉迷得不想再起来的人。

在黑客们自己编纂的用语辞典《jargon file》中,对黑客有如下定义。很长吧?仅由此便可知黑客们对于“hacker”这个词有着多么执着。

\begin{itemize}
\item 一个热衷于探索可编程系统的细节以及如何扩展它们的功能的人,他/她/它与绝大多数用户不同,后者只想学会他们工作中所需要的最低限度的东西。
\item 热衷于(甚至像是鬼上了身似的)编程本身,而不是把编程弄成一套理论的人。
\item 能够接受黑客价值(hack value)的人。
\item 一个擅长很快写出程序的人。
\item 某个特定程序的专家,以及经常使用它进行工作的人。例:“UNIX Hacker”。
\item 任何领域的专家或狂热的爱好者。例:“天文Hacker”。
\item 热衷于“用创造性克服或回避极限”这样一种智力游戏的人。
\item (误用)四处游荡、探索机密信息的恶意干涉者。比如口令Hacker、网络Hacker。正确的表达是craker。
\end{itemize}



在其他人看来,黑客是化“看似不可能”为“可能”的人。在对编程缺乏了解或是知之甚少的人看来,黑客就像是魔法师。在“沉浸于自己喜欢的事物”这一点上,黑客是幸福的;但黑客中有不少,都和“财富”“资产”这样的词绝缘。但是,世上也有黑客属于例外,他们发现了把自己的才能变为财富的方法。

\subsection{好黑客·坏黑客}
做侵入系统那样坏事的人之所以被称之为黑客,是因为那样的事也曾是“有创造性的智力挑战”,因而有黑客去做。
和一般人相比,绝大多数黑客的伦理观都显得有些轻微的错位。“有时候,比起遵守法律做个良民,满足知性的好奇心显得更为重要”这种心情,不是不能理解。但是,一个人是不是黑客和他/她/它是不是坏人是完全无关的。倒不如说,如果有真正的黑客侵入了系统,那他/她/它的动机恐怕也是“想试试看”“我想证明自己做的到”而不是“从中获取经济利益”这种不正当的动机。黑客中有许多人对于金钱之类的物质追求其实不怎么在乎。

\subsection{你是黑客吗?}
那么,各位读者,你是黑客吗?既然都点到这个网站上来了(抱歉了,ITmedia),你肯定是喜欢计算机,对编程也有兴趣吧?此外,如果你不满足于只做一个“普通的电脑用户”(那样的人肯定读的是《重新找回Windows8的开始菜单》之类的东西),而是使用具备黑客特质的类UNIX操作系统的话,那么你是黑客的可能性可以说是十分的高啦。

根据我的假说,是否拥有黑客特质,取决于是否满足以下两个条件:

首先是“你有创造欲吗?”这个问题。具有狂热特质的人大体趋于分为两类,“收藏爱好者”和“创造爱好者”。当然,也有人同时具有两种特质,这时问题就取决于在编程上你的哪种特质显现得更多。收藏爱好者关心机器的配置和新旧、以及计算机和软件的使用方法,但创造爱好者则是像“没有的话我自己造”“这里不给力,我要改改”这样,努力的改变着“世界”。所以,他们喜欢的是能够按自己的喜好进行个性化的软件,特别是有可编程性的软件。对于做东西的热情,这难道不是黑客特质的第一要素吗?

其次是“你是否做事‘刹不住车’?”。这个说法有点奇怪,不过我所知道的黑客,几乎都在某方面“刹不住车”。普通人遇到了一个觉得难以解决的问题想放弃的时候,黑客却会想再接着试试。黑客继续做下去的理由,有时是无知,有时是无谋、自信过剩,也有时是拥有超出常人的能力;但总之他们比常人更不愿放弃。所以他们才能成他人所不能成之事——当然也有时候会失败就是了。不过,黑客预备役们也都是重复着不为人知的失败才成长为黑客的。

\subsection{黑客的三大美德}
如今无人不知的脚本语言——Perl的作者,Larry Wall是当代一流的黑客。他说,程序员有三大美德:懒惰、傲慢、缺乏耐性\footnote{PS:Larry Wall对三大美德的定义

懒惰(laziness)
驱使你极力努力以减少精力的总的消耗的美德。为了节省劳力而开发的程序在被他人所使用后,为了不至于一一回答他人在使用程序时的问题而开始编写文档。因此,懒惰是程序员最重要的美德。此外也正是由于这一点,才有了本书(《Programming Perl》)的存在。

缺乏耐性(impatience)
发现计算机没干事情时候的愤怒。这成为编写不仅能够对你的指令进行回应、而且还能预测实际的指令(至少看上去是如此)的程序的原动力。因此,这是对于程序员第二重要的美德。

傲慢(Hubris)
如同碰到了宙斯的怒火一般(?),自尊心很高。这是编写(和维护)不为他人所批评的程序的原动力。因此,这是对于程序员第三重要的美德。
(《Programming Perl》,O'reilly Japan)}。当然啦,这里的“程序猿”指的是具备黑客特质的人。
可是,“懒惰”也好,“傲慢”也好,“缺乏耐性”也好,这些怎么看都像是描述缺点的词汇怎么成了“美德”了呢?Larry自己则是给这些词下了如下的定义。“为了让事情变得轻松一点而不辞辛苦”,这话听上去奇怪;不过虽然黑客对于无法满足自己求知欲的事情,连一根指头也不想动,但对于自己想做的事情,往往再多的辛劳也不以为意。此外,由于傲慢和缺乏耐性的缘故,黑客讨厌被计算机牵着走。对于生成导入计算机的数据这样的重复机械作业,他们会编写脚本,把机械作业留给计算机,或者开始开发能够一劳永逸解决此类作业的工具。在一般人看来,这也许是本末倒置的行为。但是,黑客就是这样的人。

我想在这个连载中考察这样的黑客们的生态和心理。黑客的生活方式或许能让你成为更好的程序员——不过我也打不了包票。



\end{document}
